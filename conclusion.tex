what we found:

tbd


remains unknown:

challenge the power-law mass distribution function

combine multiple spacecraft

nanodust

interstellar dust

a package of cnn dust detection







% The exhaust of particles and heat in the boundary of contemporary magnetic confinement experiments remains to this day one of the biggest challenges on the road to commercially viable fusion energy production. Due to the complexity of the physics involved in the boundary of fusion devices, the scientific community relies increasingly on numerical simulations. This approach requires a validation metric for boundary turbulence simulations such as the Filtered Poisson Process (FPP), a model which is capable to predict all relevant statistical properties of fluctuations in the scrape-off layer (SOL). In this thesis, two models for boundary plasmas are analyzed in detail. The idealized interchange model, one of the simplest models used in the past, has shown to only reproduce some of the statistical properties observed in experimental measurements. The reduced two-fluid model has proven to reproduce all relevant properties of the FPP model and experimental measurements. These results are especially encouraging when considering the assumptions and simplifications of this model, such as the reduction to two dimensions, cold ions and isothermal electrons. In addition, this thesis provides a systematical study of plasma filament interaction, concluding that studies of isolated filaments adequately describe filament motion in turbulent SOL plasmas. This thesis thereby displays the relevance and importance of numerical simulations of reduced two-fluid models for gaining a better understanding of the intricate physics of boundary plasmas. 

% Based on the presented results, a number of ideas for future work can be proposed. One next step would be the analysis of three-dimensional turbulence simulations utilizing the FPP model, as the restriction to two dimensions remains arguably the strongest simplification of the presented simulations. Recent work studying the dimensionality of SOL turbulence utilizing the STORM code, the same code basis as used in Paper III and IV, provides a useful starting point for this investigation \cite{nicholas_dim,Nicholas2021comparing}. The short time durations of 3D codes due to their computational costs remain a limitation for this project. However, this limitation could be compensated by placing several measurement probes at different binormal/poloidal positions, where we expect the statistical properties of the fluctuation time series to be statistically identical. Including additional physical parameters such as evolving electron and ion temperature, using non-Boussinesq models or a more realistic magnetic geometry may also provide additional insight. 

% Hitherto, we have only considered models derived from the Braginskii fluid equations, leaving out codes utilizing different models that are commonly used for SOL plasmas. An obvious next candidate for stochastic validation would be fully self-consistent global gyrofluid models \cite{madsen2013full,wiesenberger2014gyrofluid,wiesenberger2014radial,held2016influence,held2018non}. These models incorporate high fluctuation amplitude levels and finite Larmor radius effects and are considered to be a more complete description of the physical mechanisms in the SOL.

% As for the reduced two-fluid model, a systematical parameter scan should be performed in order to identify the model variables relevant for the moments and fluctuation statistics. Unpublished work on this topic has found a non-trivial relationship between the sheath dissipation coefficient and the scale length of the radial profile. The blob tracking algorithm presented in Paper IV might provide further insight as the blob parameters could be compared to the FPP framework for density profiles discussed in chapter \ref{n_prof}.

% The role of neutral particles in the SOL has been studied in various turbulence codes
% \cite{russell2011comparison,marandet2013influence,thrysoe2018plasma,bisai2018influence}. To this day however, no attempt has been made to analyze the effects of local ionization and recombination on the fluctuation statistics. Applying the FPP framework on one of the existing models incorporating plasma-neutral interactions would be another interesting extension of the presented work. 

% Lastly, the stochastic model of multiple seeded filaments discussed in Paper IV opens the door to a number of applications. Since it enables to define all filament parameters it provides a perfect tool to bridge the gap between isolated filaments and turbulence simulations. One example would be the evaluation of the FPP framework for density profiles in a controlled environment if turbulence simulations prove to be too demanding. 