This thesis has shown that by statistical analysis, physically meaningful parameters of the~interplanetary dust cloud can be yielded from the~dust impact counts recorded with spacecraft, which improves our understanding of the~inner solar system dust cloud. We showed that a~convolutional neural network is a~viable tool for dust identification in antenna measurements, which we used for SolO data, and which has the~potential to improve the~quality of data products of other spacecraft. We presented a~Bayesian approach to modelling the~dust counts recorded with spacecraft, which can fit more sophisticated models. We also presented and demonstrated a~statistical toolbox for this task, and we used it to characterize the~$\beta$-meteoroid dynamics. As a~part of this thesis, the~method of dust detection with antennas was studied with promising results. Understanding of the~data relies on robust knowledge of the~impact process, which we also contributed to as a~part of this thesis. Using the~formalism of kinetic theory, we explained several features of the~dust flux observed with PSP and we showed that such modelling effort is fruitful and potentially practical for other spacecraft and other dust environments. To understand the~inner solar system dust environment more deeply, and to make the~most of the~presented methods, several tasks remain for future inquiry. 

The~dust identification method presented in Paper I is significantly more reliable than the~methods used before, but it is limited to certain measurement regimes of the~specific device used on SolO. The~development of a~routine for automatic classification of waveform signals is challenging, as the~compatibility between different measurements is limited, due to differences in physical design, and in the~data products. Since human experts can classify waveforms from different devices without technical knowledge of the~device, it is feasible. Having such a~routine would be useful, as it would provide another layer of harmonization of data between different spacecraft. The~application of such routine onboard spacecraft would save the~data transfer, and therefore, potentially allow for better data coverage. 

The~statistical analysis presented in the~thesis and used in Paper II is superior to the~often used least squares fitting, since it treats the~counting error correctly. This is especially important if the~number of detections within a~temporal interval is a~small number, which it often is. It is strongly suggested that such a~method is used for future analysis of dust counts of not only SolO and PSP. 

The~assumption of a~dust flux proportional to the~impact speed to a~power higher than one was used in several works, including Paper IV of this thesis. The~reason is that since higher impact speed generates more charge, small grains are detected at a~high impact speed, but not at a~low impact speed. The~flux therefore depends on the~mass distribution of the~grains, and on the~charge production as a~function of the~impact speed. As a~simplification, it was assumed that the~produced charge depends on the~power of the~impact speed, and that the~mass distribution is a~power-law. Both are arguable. The~charge production was never experimentally measured at a~speed as high as is the~typical impact speed on PSP, or even SolO, so we are limited to a~reasonable extrapolation. More questionable is the~assumption of a~power-law distribution of masses. While it might be true for large masses, when grain to grain collisions shape the~cloud, the~dynamics of sub-micron dust depends on the~size in a~different way. This is clearly demonstrated with $\beta$-meteoroids, which move differently from bound dust, but they occupy only an order of magnitude or two on the~mass scale. Since nearly all grains within the~mass range of $\beta$-meteoroids are $\beta$-meteoroids, these are missing in the~mass distribution of bound dust. Therefore, the~mass distribution of micron and sub-micron sized bound dust grains cannot be a~power-law. A~further investigation of their distribution would prove instrumental for modelling efforts. 

Interstellar dust (ISD) was barely addressed in the~thesis, since neither SolO nor PSP are very well suited for its detection. The~situation might change when SolO's orbit becomes more inclined, and gets to the~region, where bound dust and $\beta$-meteoroids are less plentiful. This is coincidentally planned for the~late 2020's, when the~ISD flux will likely have recovered, due to the~orientation of the~solar magnetic field. Understanding the~flux observed out of the~plane of ecliptic would require generalization of the~models for the~flux but will place further constraints on the~dust parameters. We offered tools which are suitable for the~forward modelling of ISD in Paper IV and for statistical analysis of ISD in Paper II.

Several spacecraft have reported nanodust observations and since nanodust dynamics are strongly influenced by the~electromagnetic field, its flux is likely to depend on the~solar cycle. We did not observe nanodust with SolO nor with PSP, but this might be due to solar cycle. As was shown in Paper III, antenna dust detection is specific for each spacecraft, and the~understanding of the~antenna detection process is still limited. However, SolO's electrical suite is similar to that of STEREO, which detected nanodust, so nanodust detection remains an option for SolO in the~future.  

Each spacecraft's in-situ detections happen along its orbit. An intrinsic bond exists between the~velocity and the~location, and, therefore, between the~amount of detected bound dust and $\beta$-meteoroids. Spacecraft, which change their orbital elements due to gravity assists, such as SolO and PSP, change this bond in discrete steps, allowing for a~decoupling the~two components of the~flux from each other. Multi-spacecraft analysis allows for even more, as time and location are not bonded together, allowing for the~estimates of time-evolution of the~dust cloud. As demonstrated in Paper IV, multi-spacecraft analysis is complicated, but feasible. It is therefore worthy of future pursuit, as SolO will get inclined, and more dust-detecting spacecraft will operate in the~solar system simultaneously. 