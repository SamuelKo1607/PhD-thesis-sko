We demonstrated that physically meaningful parameters of the interplanetary dust cloud can be yielded from statistical analysis of dust impact counts recorded with spacecraft. In this thesis, an approach to such statistical modelling is presented, complete with characterization of the dust grains in question, the dust populations that make up the interplanetary dust cloud and unknown parameters of these, and a statistical toolbox with demonstrations of the relevant tools. The method of dust detection with antennas is accomplished and promising, but relies on robust knowledge of the impact process, contributions to which are also a part of the thesis. To make this method more fruitful, several tasks remain for future inquiry.

The dust identification method presented in Paper I is superior to the methods used before, but it is limited to certain measurement regimes of a specific device used on SolO. A development of a routine for automatic classification of waveform signals is challenging, as the compatibility between different measurements is limited, due to difference in physical design, and in the data products. Since human experts can classify waveforms from different devices without technical knowledge of the device, it is clearly not unfeasible. Having such routine would surely prove useful, as it would provide another layer of harmonization of data between different spacecraft. Deployment of such routine onboard spacecraft would greatly save the data transfer, and therefore, potentially improve the temporal coverage of the measurements as well.

Interstellar dust (ISD) was barely addressed in the thesis, since neither SolO nor PSP are very well suited for its detection. The situation might change when SolO is inclined, and gets to the region, where bound dust and $\beta$-meteoroids are less plentiful, which is coincidentally going to happen in late 2020's, when the ISD flux is likely going to recover due to the orientation of the solar magnetic field. Understanding the flux, which is going to be observed out of the plane of ecliptics, is going to be challenging, but place further constraints on the dust parameters.

Several spacecraft have reported nanodust observations and there is a general consensus that the flux of nanodust is dependent on the solar cycle. The dust data of SolO and PSP do not imply presence of nanodust, for now. SolO's electrical suite specifically is very similar to that of STEREO, which likely detected nanodust. It remains a question, whether SolO will detect appreciable amount of nonodust in the future, as the magnetic field of the Sun changes.

It was claimed in several works, including this one, that the recorded flux should be proportional to the impact speed to a power higher than one, since the higher is the impact speed, the higher is the amount of generated charge, and, by extension, smaller grains are detected, which would have been lost in noise, if the impact speed was lower. The proportionality was also rightfully claimed to depend on the mass distribution of the grains, and on the dependence of the amount of the produced charge as a function of the impact speed. As a simplification, it was claimed that the produced charge depends on a power of the impact speed, and that the mass distribution is a power-law. Both of these are arguable. The charge production was never experimentally measured at the speed as high as is the typical impact speed on PSP, or even SolO, so we are limited to a reasonable extrapolation. More questionable is the assumption of a power-law distribution of masses. While it holds true for large masses, the dynamics of sub-micron dust is dependent on the size. This is clearly demonstrated with $\beta$-meteoroids, which move differently from bound dust, but they occupy only a decade or two on the mass scale. Even so, they were assumed to be power-law distributed before by the author of this thesis and by others. If nearly all grains within the mass range of $\beta$-meteoroids are $\beta$-metoroids, the mass distribution of bound dust can clearly not be a power-law in the whole interval of detected masses. A further investigation of the mass distributions of micron and sub-micron sized dust would be beneficial for future modelling efforts.

Each spacecraft's in-situ detections happen along the orbit, limiting the reach on a single spacecraft. Intrinsic bond exists between velocity and location, and, therefore, between the amount of detected bound dust and $\beta$-meteoroids, constraining the rank of the information to 1D. Spacecraft, which migrate through the configuration space due to gravity assists, such as SolO and PSP, change this bond discretely, possibly allowing for decoupling the two components of the flux from each other. Multi-spacecraft analysis allows for even more, as time and location are not bonded together, possibly allowing for detection of time-evolution of the dust cloud. As was demonstrated, multi-spacecraft analysis is complicated, but feasible. It is therefore worthy of future pursuit, as SolO will get inclined, and more dust-detecting spacecraft will operate in the solar system simultaneously. 