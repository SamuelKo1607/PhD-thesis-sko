The~thesis aims to infer physical properties of the~interplanetary dust in terms of dust populations by analyzing experimental data. We use the~electrical antenna measurements of two spacecraft: SolO and PSP. To yield the~most information, the~data had to be treated carefully, keeping the~limited confidence of dust identification in mind. The~dust cloud analysis is indirect, as the~antenna measurements yield little information about each individual impact. Therefore, the~physical properties of the~dust populations only emerge in statistics. 

The~main body of this work is the~four papers presented below. Each of the~papers has several co-authors, and the~thesis author's particular contribution is explicitly described in the~acknowledgments in each of the~papers.

\subsubsection{Paper~I}

SolO's antennas record electrical time-domain waveforms when the~electric field shows signs of potentially interesting activity, not necessarily of dust origin. It is a~long-standing issue to convincingly classify electrical signatures by means other than visual event-by-event human labelling. In Paper~I, we report on the~development of a~machine learning tool, which provides this functionality. We tried two machine learning approaches and improvement was attained with a~purpose-developed convolutional neural network, which achieved $\SI{96}{\%}$ classification accuracy and $\SI{94}{\%}$ precision, compared to $\SI{85}{\%}$ accuracy and $\SI{75}{\%}$ precision of the~previously used algorithm. Developing this tool was instrumental for the~future work on SolO dust counts, as a~solid data product was key for any more refined analysis, such as that done in Papers II and III. 

\subsubsection{Paper~II}

There are many degrees of freedom in dust flux models. To constrain them as much as possible, we employed a~Bayesian approach to analyze the~antenna dust counts for the~first time in Paper~II. We constructed a~semi-empirical two-component model, inspired by previous works of other authors, and applied it on SolO data, specifically the~data product of Paper~I, from between 07/2020 and 12/2021, all recorded between $\SI{0.5}{AU}$ and $\SI{1}{AU}$. We estimated the~speed of outgoing $\beta$-meteoroids to $63 \pm 7 \, \si{kms^{-1}}$. We found with confidence that the~$\beta$-meteoroid population is decelerating on its way out of the~inner solar system, which clearly implies that the~radiation pressure is lower than gravity for these grains, that is $\beta < 1$. 

\subsubsection{Paper~III}

In Paper~II, we only used the~impact counts, that is the~binary information, whether an impact happened, for each of the~many temporal intervals. Electrical antenna measurements also provide information about the~amplitude and shape of each of the~impacts, which was disregarded in Paper~II. This was the~motivation for Paper~III, where we report on the~finding that the~signals recorded with SolO/RPW are typically double peaked. In fact, we found that the~chronologically first peak, denoted \textit{primary}, is explainable by the~current theory of the~dust impact signal formation. The~\textit{secondary} peak, which was found to be much more variable, was found to appear on a~significantly longer time scale explained by the~ion motion. This is possibly explained by the~escaping ions having influence on the~individual antennas. We found this secondary peak to be too strong for direct ion detection. A~possible explanation was found in the~application of the~\citet{pantellini2012nano} process, which predicts a~strong response of cylindrical antennas to the~vicinity of ions, which prevents the~photoelectron recollection for a~short time. With our adaptation of the~Pantellini process, we partially explained the~relation between the~primary and the~secondary peak's amplitude. Based on the~findings of Paper~III, we suggest that the~maximum amplitude of a~signal is not a~good proxy for the~impact-generated charge, and the~amplitude of the~primary peak is to be studied instead. 

\subsubsection{Paper~IV}

PSP detects most of the~dust impacts in the~near vicinity of the~Sun, where an important, or even dominant, portion of the~impacts is attributable to bound dust. This makes the~measurements unique with respect to other instruments, and this motivated Paper~IV, in which we: first compared the~measurements of PSP and SolO near $\SI{1}{AU}$ and, second developed a~phase-space distribution function based model for the~impact counts, which takes into account orbital parameters of the~bound dust cloud and semi-empirical parameters of the~experiment. We did not fit the~PSP data with the~model directly, as the~model is clearly too crude to replicate the~experiment, but two general features were studied. We compared the~heliocentric dependence of the~dust count predicted by the~model with the~dependence observed in the~data. We found that the~dependence of the~observed count on the~relative speed between dust and PSP is lower than previously assumed, implying a~comparatively flat mass distribution of dust inside $\SI{0.5}{AU}$. By studying the~predicted and the~observed dust count near perihelia, we found that the~flux minimum observed near the~perihelia is too prominent to be explained by the~alignment between the~spacecraft velocity and the~dust velocity. We offered alternative explanations for the~flux minimum. 