The focus of the thesis is on deriving physical properties of the interplanetary dust in terms of dust population from the data of two unique experiments: Solar Orbiter and Parker Solar Probe. To yield the most information, the data had to be treated carefully, with numerous limitations kept in mind. The analysis of the dust cloud is very indirect, as both experiments yield very little information about each individual impact. Therefore, the physical properties of the dust populations only emerge in statistics. 

\subsubsection{Paper I}

Solar Orbiter's antennas records electrical time-domain waveforms when the electric field shows signs of potentially interesting activity, not necessarily of dust origin. It's a long standing issue to convincingly classify the electrical signatures by means other than visual event-by-event human labelling. The goal of Paper I was to develop a machine learning tool, which would provide this functionality. Two machine learning approaches were tried and a success was achieved with a purpose-developed convolutional neural network of $\SI{96}{\%}$ classification accuracy and $\SI{94}{\%}$ precision, compared to $\SI{85}{\%}$ accuracy and $\SI{75}{\%}$ precision of the previously used algorithm. Developing this tool was instrumental for the future work on Solar Orbiter dust counts, as a solid data product was key for any more refined analysis, such as that done in Papers II and III. 

\subsubsection{Paper II}

There are many degrees of freedom in reasonable dust flux models. To constrain them as much as possible, Bayesian approach was taken in Paper II to analyze the data product of Paper I. A semi-empirical two-component model, greatly inspired by previous work of other authors, was constructed and applied on Solar Orbiter data from between 07/2020 and 12/2021, all recorded between $\SI{0.5}{AU}$ and $\SI{1}{AU}$. Previous estimates of $\beta$ meteoroid speed by other authors of about $\SI{50}{kms^{-1}}$ were refined to $\SI{63 \pm 7}{kms^{-1}}$. It was found with confidence that the $\beta$ meteoroid population is decelerating on its way out of the inner solar system, which clearly implies $\beta < 1$. 

\subsubsection{Paper III}

Although the amount of information made available by electrical antenna measurements is a lot lower than in the case of a dedicated detector, there is physical information to be harnessed, beyond the information that an impact occurred. This was the motivation for Paper III, in which we report on the finding that the signals recorded with SolO/RPW are typically double-peaked. In fact, we found that the chronologically first peak, denoted \textit{primary}, is explainable by the current theory on the dust impact signal formation. The \textit{secondary} peak, which was found to be much more variable, was found to appear on a significantly longer time scale of ion motion, which is possibly explained by the influence of antennas by ions. This peak was found to be too strong for a direct detection, and a possible explanation was found in the application of so-called Pantellini process of photoelectron feedback. Based on the findings of this article, we suggest that maximum amplitudes of the signals are not a good proxy for the amount of impact-generated charge, and the amplitudes of the primary peaks are to be studied instead.

\subsubsection{Paper IV}

PSP detects most of the dust impacts in the near vicinity of the Sun, where an important or even dominant portion of the impacts is attributable to bound dust. This makes the measurements unique with respect to other instruments, and motivates Paper IV, in which we developed a phase-space distribution function based model for the impact counts, which takes into account orbital parameters of the bound dust cloud and semi-empirical parameters of the experiment. This model is not directly applied to the data, as the model is clearly too crude to replicate the experiment, but two general features are studied. First, through comparing the heliocentric dependence of the dust count predicted by the model with the dependence observed in the data, we found that the dependence of the observed count on the relative speed between dust and PSP is likely lower than previously assumed. By studying the predicted and the observed dust count near perihelia, we found that the dip observed near the perihelia is too deep to be explained by the alignment between the spacecraft velocity and the dust velocity. We offered alternative explanations for the dip. 