The main focus of this thesis is on investigating the statistical properties of fluctuations in numerical simulations of SOL plasmas.  Papers I and III study time series of single point measurements of established fluid models utilizing the FPP framework. Paper II provides additional theoretical insight to the results obtained in Paper I. Lastly, Paper IV investigates the interaction of filaments in SOL simulations and thereby discusses to what degree filaments can be considered uncorrelated, which is a basic assumption for the FPP model. The papers are ordered thematically.

In order to investigate the statistical properties of fluctuations described by different SOL models, we start with the idealized interchange model discussed in Paper I, arguably the simplest self-consistent description of SOL turbulence. Fluctuation time series at different radial positions are obtained and analyzed by stochastic modeling. We observe that the PDFs for the temperature fluctuations change from a normal distribution in the center of the simulation domain to a distribution with an exponential tail at the boundary of the simulation domain, a result consistent with experimental measurements of SOL plasmas. The PSDs have an exponential shape, which can be attributed to the underlying Lorentzian pulses, identified by a deconvolution method. The time series of the temperature show periods of strongly intermittent fluctuations with large bursts, interrupted by quiescent periods with quasi-periodic oscillations. These alternating periods can be attributed to the generation of a sheared mean flow through the fluid layer resulting in predator-prey-like dynamics of the kinetic energy integrals. Since this behavior has not been observed in time series of experimental measurements, it becomes clear that the utilized Rayleigh-Bénard-like model is insufficient to reproduce all statistical properties of SOL fluctuations.

Paper II provides additional theoretical insight as the shot noise process with periodic arrivals is investigated. It is shown analytically that the PSD of a shot noise process with periodic arrivals, Lorentzian pulses and exponentially distributed amplitudes has an exponentially modulated Dirac comb of decaying amplitudes. In addition, we provide numerical realizations of a shot noise process with quasi-periodic arrival times by using a narrow, uniform distribution for each arrival around the strictly periodic arrival time. We find that moderate deviations from perfect periodicity destroys the Dirac comb as it leads to a broadening of its peaks and the decrease of the peak amplitudes for higher harmonics. The resulting PSD resembles the findings of Paper I, thereby confirming that the statistical properties of the fluctuation time series of the idealized interchange model are due to the presence of quasi-periodic Lorentzian pulses.

In Paper III we study the statistical properties of reduced two-fluid models and thereby increase the model complexity and number of considered physical effects compared to the idealized interchange equations. Again, we observe that the PDFs of fluctuation time series of the plasma density show an exponential tail in the far SOL. In contrast to the model discussed in Paper I, we find that the average burst or pulse shape is well described by a two-sided
exponential function. The PSD of the particle density is that of the average pulse shape and does not change with radial position. The amplitudes and the waiting times between two consecutive arrivals are exponentially distributed. The profiles have an exponential form with radially constant scale length. As for the moments, we find that the fluctuation level increases with radial position and a nearly parabolic relationship between skewness and flatness moments. All of these results stand in perfect agreement with the predictions of the FPP model. In contrast to experimental measurements, we can choose an arbitrarily high sampling frequency for the single point measurements in these simulations. For frequencies higher than what is experimentally feasible, we observe an exponential spectrum in the PSD and a continuous, Lorenztian-like peak in the averaged pulse shape. These results cannot be compared directly to experimental counterparts due to the poor sampling rate by the diagnostics. 

The last paper included in this thesis investigates the interaction of single filaments with each other and thereby addresses the question to what extent filaments can be considered isolated. As the FPP model assumes all pulses in a time series to be uncorrelated, this study remains highly relevant for the work presented in Papers I-III. A reduced two-fluid model is used for this investigation. In order to track filaments and determine filament parameters, a blob tracking algorithm based on an amplitude threshold method is presented. The velocity estimates of the algorithm are validated by a conventional center of mass approach. We introduce a model of multiple seeded filaments where the filament parameters, i.e. size, initial position, amplitude and arrival time, are sampled from appropriate distribution functions. A model-specific intermittency parameter is introduced which quantifies the level of filament interactions as a function of their average size, velocity and waiting time. This model is then studied for different levels of complexity and filament interaction compared to theoretical size-velocity scaling laws of perfectly isolated filaments. We observe an increase in the average radial velocity for strongly interacting filaments. This is found to be caused by the interaction of filaments with the electrostatic potential of one another. The blob tracking approach is then applied on full plasma turbulence simulations where a strong correlation is found between filament amplitudes, sizes and velocities. Despite the observed increase in the radial velocities for strongly interacting filaments, we observe a systematical size-velocity relationship consistent with theoretical predictions. We therefore conclude that filaments can be regarded to lowest order as isolated structures, i.e., that the corresponding pulses in the FPP model can be assumed to be uncorrelated. 