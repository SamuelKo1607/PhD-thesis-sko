The focus of the thesis is on deriving physical properties of the interplanetary dust in terms of dust population from the data of two unique experiments: Solar Orbiter and Parker Solar Probe. To yield the most information, the data had to be treated carefully, with numerous limitations kept in mind. The analysis of the dust cloud is very indirect, as both experiments yield very little information about each individual impact. Therefore, the physical properties of the dust populations only emerge in statistics. 

\subsubsection{Paper I}

Solar Orbiter's antennas records electrical time-domain waveforms when the electric field shows signs of potentially interesting activity, not necessarily of dust origin. It's a long standing issue to convincingly classify the electrical signatures by means other than visual event-by-event human labelling. The goal of Paper I was to develop a machine learning tool, which would provide this functionality. Two machine learning approaches were tried and a success was achieved with a purpose-developed convolutional neural network of $\SI{96}{\%}$ classification accuracy and $\SI{94}{\%}$ precision, compared to $\SI{85}{\%}$ accuracy and $\SI{75}{\%}$ precision of the previously used algorithm. Developing this tool was instrumental for the future work on Solar Orbiter dust counts, as a solid data product was key for any more refined analysis, such as that done in Papers II and III. 

\subsubsection{Paper II}

There are many degrees of freedom in reasonable dust flux models. To constrain them as much as possible, Bayesian approach was taken in Paper II to analyze the data product of Paper I. A semi-empirical two-component model, greatly inspired by previous work of other authors, was constructed and applied on Solar Orbiter data from between 07/2020 and 12/2021, all recorded between $\SI{0.5}{AU}$ and $\SI{1}{AU}$. Previous estimates of $\beta$ meteoroid speed by other authors of about $\SI{50}{kms^{-1}}$ were refined to $\SI{63 \pm 7}{kms^{-1}}$. It was found with confidence that the $\beta$ meteoroid population is decelerating on its way out of the inner solar system, which clearly implies $\beta < 1$. 

\subsubsection{Paper III}



\subsubsection{Paper IV}