The~inner solar system dust cloud is a~dynamic environment, located between the~inner planets and the~Sun. Some of the~open questions about the~dust cloud are difficult to address with remote sensing instruments, so in-situ detection, that is the~local detection from within the~dust cloud, is essential. Two recent and unique missions, Solar Orbiter (SolO) and Parker Solar Probe (PSP) venture as close to the~Sun as $\SI{0.28}{AU}$ and $\SI{0.06}{AU}$ respectively, and both are suited for the~detection of dust impact induced charge cloud with their electrical antennas.   

Since electrical antennas are not purpose-built dust detectors, identifying dust impacts and interpreting the~signals is ambiguous and a~great deal of attention is paid to these tasks. To address the~identification issue, we developed a~machine learning, convolutional neural network (CNN) detection routine, which is superior to the~previously used identification algorithms with its $\SI{96}{\%}$ accuracy and $\SI{94}{\%}$ precision. This routine is applied on several years of SolO's Radio and Plasma Waves (RPW) time domain sampled (TDS) electrical waveforms, providing the~highest quality dust data product available for SolO. To address the~signal interpretation issue, we present a~study of thousands of SolO/RPW/TDS waveforms, which were inspected in detail, to understand the~non-conventional double-peaked signature. The~signature is at least partially explained with an interaction between the~impact generated charge and photoelectron sheath of the~electrical antennas. This explanation provides a~sharper measure for the~total amount of impact generated charge, and for the~post-impact speed of the~charge cloud. On top of that, the~impact location on the~SolO's body is studied and the~data is consistent with most impacts happening on the~heat shield and on the~ram side of the~spacecraft.  

The~dust grain motion depends on the~grains' properties. When the~motion of the~grains is understood, properties of the~grains are constrained. This is made more difficult if several dust populations are present simultaneously, which differ by their size, origin, speed, and lifetime. The~modelling approach must take the~populations into account, and we demonstrate a~tool suitable for this task. By assuming a~two-component dust cloud observed by SolO and with Bayesian fitting of the~Poisson-distributed detection counts, we confirmed that the~data is indeed consistent with most of the~detections being due to $\beta$-meteoroids with $\beta\gtrsim 0.5$, as the~data is only consistent with decelerating $\beta$-meteoroids. Unlike SolO, PSP experiences mostly bound dust impacts in a~portion of its orbit. This allows for a~study of a~single component dust flux. We developed a~model for bound dust impact counts using the~formalism of the~phase space distribution function, known from kinetic theory. The model considers several important dust cloud parameters which were not used previously. Using the~model, we found that the~observed flux is only compatible with a~differential mass distribution slope $\delta$ between $0.14$ and $0.49$, shallower than previously reported for bound dust. We therefore question the~validity of the~assumption of power-law distributed masses of the~$\si{\mu m}$-sized bound dust grains near the~Sun. We found that the~flux minimum observed in each perihelion is too prominent to be explained by the~alignment between the~velocity of bound dust and the~spacecraft, especially when the~eccentricities, inclinations and other poorly constrained parameters of the~dust grains are considered. Although PSP and SolO have different orbits, we were able to see indication that the~PSP's heat shield is less sensitive to dust impacts, by comparing the~PSP and SolO data close to $\SI{1}{AU}$. With Bayesian modelling of dust impacts on SolO and on PSP together, we were able to estimate that the~PSP's heat shield is by a~factor of four less sensitive to the~dust impacts, compared to the~rest of PSP and SolO.  

The~tools presented in this thesis, specifically the~CNN dust identification algorithm, and the~phase space distribution function dust model are suitable for future application with other spacecraft. Together with the~Bayesian flux fitting approach, these tools allow for more precise modelling and therefore deeper understanding of in-situ dust detection.  