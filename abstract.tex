Inner solar system dust environment is a dynamic dust system, which sits between the inner planets and the solar corona. Questions remain open about the inner solar system dust environment: what are the sources of the dust in the cloud? What is the mass distribution of the grains, and how to their properties depend on their mass? What it the composition of these grains? Where does the near solar dust depletion zone start and what does it look like? 

Some of the questions are hardly addressed with a remote sensing instrument, so in-situ detection is essential. Two recent missions, Solar Orbiter (SolO) and Parker Solar Probe (PSP) venture as close to the Sun as $\SI{0.28}{AU}$ and $\SI{0.06}{AU}$ respectively, and both are suited for the detection of dust impact induced charge cloud with their respective electrical antennas. 

Since electrical antennas are not purpose-built dust detectors, identifying dust impacts and interpreting the signals is not unambiguous and a great deal of attention is paid to these tasks. On the former, we report on a successful development of a machine learning, convolutional neural network (CNN) detection routine, which is greatly superior to the previously used identification algorithms with its $\SI{96}{\%}$ accuracy and $\SI{94}{\%}$ precision. This routine is applied on several years of SolO's Radio and Plasma Waves (RPW) time domain sampled (TDS) electrical waveforms, providing the highest quality dust data product available for SolO. On the latter, we present a study of thousands of SolO/RPW/TDS waveforms, which were studied by eye and statistically, to understand the non-conventional double-peaked signature. The signature is at least partially explained with an interaction between the impact generated charge and photoelectron sheath of the electrical antennas. This explanation provides a sharper measure for the total amount of impact generated charge, and for the post-impact speed of the charge cloud. On top of that, the impact location on the SolO's body is studied and the data is consistent with most impacts happening on the heat shield and on the ram side of the spacecraft.

Dust grains' motion depends greatly on the grains' properties. When the motion of the grains is understood, properties of the grains are constrained. This is made more difficult by simultaneous presence of dust grains of different populations, differed by their size, origin, speed and life-time. Modeling approach must take the populations into account. By assuming a two-component dust cloud observed by SolO and with Bayesian fitting of the Poisson-distributed detection counts, we confirmed that the data is indeed consistent with most of the detections to be due to $\beta$-meteoroids with $\beta\gtrsim 0.5$, as the data is only consistent with decelerating $\beta$-meteoroids. Unlike SolO, PSP experiences mostly bound dust impacts in a portion of its orbit. This allows for a study of a single component dust flux. We model the dust density using the formalism known from plasma science, and we found that the observed flux is only compatible with the flux scaling with the impact speed to the power of $1.5 - 2.5$, posing questions about the validity of the assumption of power-law distributed masses of the the $\mu m$-sized bound dust grains. We observed that the flux minimum observed in each perihelion is too prominent to be explained by the alignment between the velocity of bound dust and the spacecraft, especially as a distribution of eccentricies, inclinations an other poorly constrained dust cloud parameters are taken into account. Although PSP and SolO have different orbits, by comparing their data close to $\SI{1}{AU}$, we were able to see indication that the PSP's heat shield, the Thermal Protection System (TPS) is less sensitive to dust impacts. With Bayesian modelling of dust impacts on SolO and on PSP together, we were able to estimate that the TPS is by a factor of four less sensitive to the dust impacts, compared to the rest of PSP and SolO. 