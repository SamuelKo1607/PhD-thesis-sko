Different sources of dust, along with different forces acting on dust grains as a~result of their location and size, allow us to distinguish {populations} of dust. For the~context, in which the~presented research is set, we are going to introduce and characterize several important dust populations in the~solar system.

\section{Bound dust}

Among the~forces discussed in Ch.~\ref{ch:forces}, the~force with the~steepest proportion to the~grain's radius $r$ is gravity: $F_g \propto r^3$. Gravity therefore determines the~motion of large bodies in the~solar system. The~dust particles that are dominantly influenced by gravity and are in bound orbits, are denoted \textit{bound dust}. The~term \textit{zodiacal dust} is sometimes used since this dust contributes to the~zodiacal light the~most. The~term \textit{F-corona} is used for the~zodiacal light observed close to the~Sun. We may encounter the~term \textit{alpha} (abv. $\alpha$) meteoroids, sometimes used interchangeably with bound dust, but appropriateness of the~term as a~synonym to bound dust was debated \citep{sommer2023alpha}, since the~term was originally coined to describe highly eccentric bound grains only \citep{zook1975source} and, therefore, we will not use the~term.

\subsubsection{Size range}

The lower size threshold for a~dust grain to be bound may be estimated by comparison with the~forces dependent $\propto r^2$, namely radiation pressure force $F_{RP}$ and Poynting-Robertson drag $F_{PR}$. If we set the~criterion for $F_{RP} \ll F_g$ to $\beta<0.1$, then using Eq.~\ref{eq:beta_estimate} we find $r>10 \, \si{\mu m}$. The~Poynting-Robertson lifetime of $r=10 \, \si{\mu m}$ dust (Eq.~\ref{eq:PR_estimate}) is approximately $13 \, \si{kyr}$, which allows for many orbits. For completeness, using Eqs.~\ref{eq:charge_estimate} and \ref{eq:EM_estimate} we find that for $r=10 \, \si{\mu m}$ dust, the~electromagnetic force $F_{EM} < 2\cdot 10^{-4} F_g$. Therefore, the~motion of $r>10 \, \si{\mu m}$ dust is dominated by gravity.

\citet{leinert1981zodiacal} found that the~range of $10 \, \si{\mu m} < r < 100 \, \si{\mu m}$ contributes to the~intensity of Zodiacal light the~most. The~reason for this is that much larger dust has a~low ratio of surface to mass, and much smaller dust neither lasts long in orbit, nor scatters light very effectively. 

\subsubsection{Dynamics}

The number density $n$ of $10 - 100 \, \si{\mu m}$ dust near the~ecliptic plane was established based on remote observations of the~zodiacal light to follow 
\begin{equation}
    n(R) \propto R^{-1.3} \label{eq:dust_number_density}
\end{equation}
in the~range of $0.3 \, \si{AU} < R < 1 \, \si{AU}$ \citep{leinert1981zodiacal}, and since the~zodiacal light seems to be constant \citep{buffington2016measurements}, it is believed to be constantly replenished by fragments of colliding, larger, $r \geq 1 \, \si{mm}$ dust with the~number density $n \propto R^{-\nu}$, where $1 < \nu < 1.1$ \citep{leinert1983maintain}.

Bound dust grains are gradually losing mass due to erosion. The~erosion is faster if the~grains are closer to the~Sun. This process is therefore accelerated by shrinking the~orbital distance due to Poynting-Robertson drag. The~grains also lose mass abruptly, in collisions, as we discussed in Sec.~\ref{ch:collisions}. It was found that collisions are the~reason for the~greatest proportion of the~mass loss from the~bound dust population \citep{grun1985collisional}. On the~other hand, the~population is refilled by dust newly released from larger bodies. Although debris of asteroidal collisions contribute to the~population of bound dust, the~majority of bound dust is likely created in fragmentation of comets. This is supported by numerical modelling of the~motion and the~collisional cascade of dust grains ejected by different orbital group bodies \citep{nesvorny2010cometary}, and by the~comparison of the~reflective properties of bound dust to those of comets and various asteroidal groups \citep{yang2015origin}. 

It was found \citep{dohnanyi1969collisional} that erosive and catastrophic collisions between asteroids (Sec.~\ref{ch:collisions}) leads to a~steady-state mass distribution of the~asteroids in the~power-law form: 
\begin{equation}
    f(m) \propto m^{-\delta}, \label{eq:mass_distribution}
\end{equation}
where $\delta = 11/6 \approx 1.83$, provided that the~crushing-law (Eq.~\ref{eq:crushing_law}) slope $\eta < 2$, that is, there are not too many large objects left after the~collision. It is important to note that under the~assumption that $\eta < 2$ and that the~catastrophic collisions dominate the~mass loss, the~slope $\delta = 11/6$ is not a~function of $\eta$ at all and is only a~very shallow function of $\eta$ if the~erosive collisions are relevant. It was also shown that the~solution of $\delta = 11/6$ is stable if perturbed by an additional inflow of low or high $m$ into the~distribution \citep{dohnanyi1969collisional}. This does not hold fully for the~dust grains with $r < \si{\mu m}$, since more loss processes are relevant in this region \citep{grun1985collisional}, most notably, $\beta$-meteoroid production.

\subsubsection{Open questions}

Remote observations provide information about the~spatial distribution of dust, while in-situ detections and modelling provide information about its mass distribution. Both are however fairly insensitive to dust eccentricities and inclinations, therefore to the~velocity distribution and the~spatial distribution out of the~ecliptic plane. Bound dust is an important contributor to the~counts observed by PSP, and the~compatibility of the~observed counts with eccentric and inclined dust was studied in Paper IV. While it is clear that the~dust is concentrated around the~ecliptic plane, it is not straightforward to deduce the~off-ecliptic density dependence from remote observations and several models were proposed and debated \citep{giese1986three}. Future measurements of SolO might shed some light on this, as the~spacecraft, which performs in-situ dust measurements, will get gradually inclined between $2025$ and $2029$.

Since the~lifetime of dust grains in near vicinity of the~Sun is short due to intense erosion processes (Sec.~\ref{ch:erosion}), a~dust-free zone (DFZ) was hypothesized \citep{russell1929meteoric}. DFZ might be enveloped by a~transition region, so called dust-depletion zone (DDZ). Due to the~limitations of the~experimental techniques and the~difficulties with decomposing the~brightness measurements to dust and other light-producing phenomena, the~DFZ was not convincingly observed to this date, although DDZ was observed remotely from onboard PSP \citep{stenborg2018characterization}. It was thus estimated, that the~DDZ lies between $5 \, R_{Sun}$ and $19 \, R_{Sun}$, while the~DFZ is expected inward of $5 \, R_{Sun}$ \citep{stenborg2022psp}. This implies that PSP has already travelled well into the~DDZ with its perihelia of $\approx 12 \, R_{Sun}$. An investigation of the~DDZ was one of the~objectives of Paper IV, results of which suggest that the~DDZ is necessary to explain the~observed dust detection counts near the~perihelia of PSP.

\section{Beta meteoroids}

Pioneer 8 and 9 spacecraft discovered a~previously unobserved population of dust grains, when it was apparent that most of the~grains of the~size $r < 10^{-6} \, \si{m}$ were coming from the~direction of the~Sun \citep{berg1973evidence}. This phenomenon is explained as a~population of dust leaving the~Sun's gravity well on a~hyperbolic trajectory due to the~solar radiation pressure successfully competing with solar gravity, and the~term \textit{beta meteoroids} ($\beta$\textit{-meteoroids}) was coined for the~population \citep{zook1975source}. Many other space missions observed the~population since, for example Solar Terrestrial Relations Observatory ({STEREO}) \citep{zaslavsky2012interplanetary}, Wind \citep{malaspina2014interplanetary}, and SolO \cite{zaslavsky2021first}.

\subsubsection{Size range}

As we showed previously (Eq.~\ref{eq:effective_gravity}), the~laws of motion do not change even if the~radiation pressure is significant, it is the~effective gravitational parameter $\mu_{e} = (1-\beta) \mu$, which changes with $\beta$. Since the~escape speed $v_e$ and the~circular speed $v_c$ differ by a~factor of $\sqrt{2}$ and both depend on $\mu$ (Eqs.~\ref{eq:circular_speed}, \ref{eq:escape_speed}), in terms of effective gravity:
\begin{equation}
    v_c(\beta=0) = v_e(\beta=1/2).
\end{equation}
This implies that if a~dust grain on a~circular orbit with $\beta=0$ suddenly changes its $\beta$ value to $\beta = 1/2$, it becomes critically unbound, that is on a~parabolic trajectory. A~change to a~higher $\beta$ would then naturally lead to a~hyperbolic trajectory. Since dust grains change their $\beta$ value suddenly at collisions, and since the~post-collision fragment speed is not going to be very different from the~pre-collision parent object speed, the~value of $\beta=1/2$ is often considered the~minimum $\beta$ required for the~dust grain to be on an unbound trajectory pointed away from the~Sun.

Using our estimate with simplistic assumptions (Eq.~\ref{eq:beta_estimate}), $\beta > 0.5$ for $r < 1.9 \, \si{\mu m}$. More refined estimates \citep{kimura2003composition} point to the~region of $100 \, \si{nm} < r < 1 \, \si{\mu m}$ and are material dependent. We note that if the~parent grains are on eccentric orbits, the~requirement of $\beta > 0.5$ is not exact, but this does not influence the~size estimate greatly. We also note that beta meteoroids are smaller than bound grains with $r>\SI{1}{\mu m}$, which seem to contribute to the~brightness of the~Zodiacal light the~most \citep{leinert1981zodiacal}. This makes remote observation difficult, and they are therefore, in practice, only reliably detected in-situ.

\subsubsection{Dynamics}

The beta meteoroids are believed to be created in collisions between bound grains, which naturally happens where the~bound dust spatial density and the~relative velocities between the~grains are high. It was recently reported that the~dust detections of PSP are compatible with the~beta meteoroid creation region at around $10 - 20 \, R_{Sun}$ \citep{szalay2021collisional}. Since beta meteoroids are unbound, they leave the~inner solar system shortly after their creation, and other forces have limited time to act. For example, even relatively short Poynting-Robertson lifetime of $\approx \, kyr$ is unimportant compared to the~timescale of $< 1 \, \si{yr}$, in which a~grain created in the~vicinity of the~Sun passes beyond $1 \, \si{AU}$. On their way out, beta meteoroids follow the~conservation of angular momentum and the~conservation of energy. The~former implies that sufficiently far from the~region of their creation, their velocity is nearly radial. The~latter implies they accelerate if $\beta > 1$, since they feel net solar repulsion, and they decelerate if $\beta < 1$, since they feel net solar attraction. In the~special case of $\beta=1$, the~grains neither accelerate nor decelerate, and their number density sufficiently far from the~Sun depends on the~heliocentric distance $R$ as
\begin{equation}
    n(R) \propto R^{-2}. \label{eq:beta_number_density}
\end{equation} 

\subsubsection{Open questions}

One of the~open questions related to beta meteoroids is to what extent their flux is constant in time and rotationally symmetric. Answering this question is complicated, since beta meteoroids are only detected in-situ, and there is always a~bond between the~time and location of the~detecting spacecraft. Beta meteoroids were claimed to be produced in collisions between the~main, rotationally symmetric bound dust population and the~{Geminids} meteoric stream \citep{szalay2021collisional}, as a~means to explain excess detections with PSP. Further investigation of this phenomenon is desirable.

The dynamics of $\beta$-meteoroids is understood theoretically, but there is little experimental evidence about where they are created, their speed, mass distribution, and other parameters, which influence their dynamics. $\beta$-meteoroids are important for both PSP and SolO observations. In Paper II of this thesis and using the~measurements of SolO, we estimated the~speed of $\beta$-meteoroids. We found that they decelerate significantly on their way out of the~inner solar system, which implies effective mean $\beta$ near the~liberation threshold, that is $\beta \approx 0.5$. We studied the~features of individual impacts in Paper III, proposing a~better measure for impact-generated charge, enabling a~more precise study of the~mass distribution of the~grains. 

\section{Interstellar dust}

We know that dust of various sizes is present in the~interstellar medium of our galaxy since it is needed to explain the~interstellar extinction measurements \citep{desert1990interstellar}. The~heliosphere moves with respect to the~local interstellar medium. The~relative speed and the~direction are known, since the~velocity distribution of interstellar neutral gas was measured, for example onboard Ulysses spacecraft \citep{witte2004kinetic}. It was also with the~Ulysses spacecraft, that a~population of dust, coming from the~direction of the~interstellar neutral gas was detected \citep{grun1993discovery}. Since then, other spacecraft reported interstellar dust ({ISD}) detections in-situ, such as Galileo \citep{baguhl1995flux}, Cassini \citep{altobelli2003cassini}, STEREO \citep{zaslavsky2012interplanetary}, and Wind \citep{malaspina2014interplanetary}. 

\subsubsection{Size range and dynamics}

ISD is created by condensation and aggregation, and destroyed by sputtering, sublimation, and collisions \citep{mann2010interstellar}. Only the~ISD grains with masses higher than $3\cdot 10^{-16} \, \si{kg}$ (that is $r \approx 0.3 \, \si{\mu m} $) are believed to be able to enter the~heliosphere \citep{kimura1998electric}. Upon entry, the~grains move according to the~laws of gravity and under the~influence of Lorenz force. The~effective gravitational parameter $\mu_e(\beta)$ (see Eq.~\ref{eq:effective_gravity}) depends on the~amount of radiation pressure compared to gravity. Should $\beta > 1$, the~grains are deflected and do not reach close vicinity of the~Sun \citep{henriksen2022interstellar}. If $\beta \approx 1$, they do not feel the~influence of the~Sun and move with nearly constant speed. The~motion of the~dust grains is also influenced by the~Lorentz force, and this effect is also size dependent \citep{morfill1979motion}.

\subsubsection{Open questions}

The exact dynamics of ISD grains in the~solar system is an object of study, especially the~possible effect of dust \textit{focusing} and \textit{defocusing} to and from the~plane of ecliptic due to the~polarity of the~interplanetary magnetic field, which switches between N-S and S-N configurations with the~period of 22 years, that is two solar cycles \citep{morfill1979motion}. This long period, longer than the~duration of many experiments, makes the~effect difficult to study. It is however the~case that most of the~in-situ ISD detections happened around the~year 2009, during the~solar minimum between solar cycles 23 and 24, when the~interplanetary magnetic field was in the~focusing configuration \citep{babic2022situ} and the~observed ISD flux decreased significantly, although not disappear completely, since then. Although instrumental explanations remain possible, it was hypothesized that the~change in observed flux is physical and that the~flux should rise again during the~minimum between solar cycles 25 and 26, which is expected in late 2020's \citep{mann2010interstellar}. 

Gravitational focusing of the~ISD grains with $\beta < 1$ and an increased spatial density in the~inner heliosphere and especially behind the~Sun is a~logical consequence of their trajectories \citep{mann2010interstellar}, yet experimental evidence of this is scarce \citep{altobelli2006new}. The~low density, the~temporal variability, and the~difficulty of distinguishing this population from other dust complicates the~detection. Neither SolO nor PSP observed clear evidence of ISD.

\section{Nanodust}

Small dust grains are produced in collisions of larger dust. If the~created dust grains are so small that $\beta \ll 1$, the~radiation pressure does not liberate them from the~gravity well of the~Sun. Small grains have, however, a~high capacitance to mass ratio $C/m$, and by extension, high charge to mass ratio $q/m$. For example, $r = 10 \, \si{nm}$ spherical grain at the~potential of $\phi = 1 \, \si{V}$ has the~charge density $q/m \approx 71 \, \si{Ckg^{-1}}$ (Eq.~\ref{eq:charge_estimate}). The~defining feature of \textit{nanodust} grains is that they are therefore highly susceptible to be influenced by Lorentz force \citep{czechowski2010formation}. 

As we discussed in Sec.~\ref{ch:solar_wind_pressure}, solar wind pressure provides an additional pseudo-Poynting-Robertson drag, which is usually smaller than radiational Poynting-Robertson drag for larger dust, but is likely very relevant for small dust with $r<10\, \si{nm}$ \citep{mukai1982solar}, when even radial solar wind pressure might contest gravity. 

Planetary nanodust was also identified in Cassini's Radio and Plasma Wave Science ({RPWS}) spectra 
at $\SI{1}{AU}$ \citep{schippers2014nanodust}, and between $\SI{1}{AU}$ and $\SI{5}{AU}$, and it was concluded, that the~asteroid belt's contribution to the~nanodust flux is negligible \citep{schippers2015nanodust}. It was also identified in the~Jovian system \citep{meyer2009detecting}, and in the~Saturnian system \citep{kempf2005high}. It was in fact concluded that nanodust is so ubiquitous, that some was detected whenever the~RPWS instrument was on \citep{schippers2015nanodust}. Cometary nanodust was detected by Giotto/PIA and Vega/PUMA mass spectrometers near the~comet Halley, although limited information about their composition was yielded, due to low signal \citep{utterback1990attogram}.

\subsubsection{Open questions}

Since nanodust grains are highly susceptible to the~influence of Lorentz force, a~strong temporal variation is naturally expected, as the~interplanetary magnetic field is not constant \citep{poppe2020effects}. The~complications are that nanodust is not observed remotely, and observation in-situ is problematic \citep{pantellini2012nano,kellogg2016dust,kellogg2017note}. 

Nanodust was reportedly detected on STEREO spacecraft \citep{meyer2009dust} and mostly disappear after solar cycle 24 started after 2010 \citep{zaslavsky2012interplanetary}. It was argued that this might be due to an unfavorable interplanetary magnetic field orientation during solar cycle 24, and that the~nanodust flux will reappear in the~STEREO measurements later during solar cycle 25, at some time before 2028 \citep{poppe2022effects}. The~presence of nanodust in the~SolO and PSP data remains a~possibility. However, the~counts can be explained with models without nanodust, therefore, it is a~minor contributor for now. This might change later during solar cycle 25. 

\section{Localized dust}

\subsubsection{Planetary dust}

The dust linked to a~planet, is called \textit{planetary dust}. The~passage of Pioneers 10 and 11 through the~Jovian system discovered flux of dust several orders of magnitude higher than the~flux commonly observed elsewhere at the~same heliocentric distance \citep{humes1974interplanetary}. The~subsequent study by Ulysses, which measured intermittent dust streams originating in the~Jovian system \citep{grun1993discovery}, and Voyager, which revealed the~active volcanism at Io \citep{kruger2004jovian}, confirmed the~Jovian system as a~locally important source of dust. Cassini's Cosmic Dust Analyzer ({CDA}) and RPWS detected nanodust near Jupiter, which was also confirmed to be originating at the~moon Io \citep{meyer2009detecting}. Similarly, in the~Saturnian system, Enceladus was linked to the~tenuous E-ring of Saturn \citep{baum1981saturn}. Cassini/CDA data also showed nanodust detections \citep{kempf2005high} and later confirmed the~volcanic activity on Enceladus \citep{spahn2006cassini}, which feeds the~E-ring of Saturn \citep{kempf2010enceladus}. 

\subsubsection{Inhomogeneity}

Unlike the~omnipresent gradual erosion, the~inflow of new dust into the~system is very stochastic and non-constant, as for example comets, which are believed to be an important source of the~dust cloud, are not uniformly distributed in time and space. More than half of all the~catalogued comets are \textit{sungrazers}, which are comets which have the~perihelion in a~close vicinity of the~Sun \citep{jones2018science}, at the~heliocentric distance of a~few solar radii $R_{Sun}$. These are typically small objects ($r<100 \, \si{m}$) which do not survive the~passage but are destroyed near the~perihelion by the~combination of heat and tidal waves. Their material is then partially transferred to the~dust cloud. In general, such highly eccentric comets might arrive from any direction, but a~few massive bodies were identified to have been destroyed in the~past, which are responsible for most of the~identified near-sun comets \citep{jones2018science}. Among these, the~most prominent cometary group is the~{Kreutz group} \citep{kreutz1888untersuchungen}, members of which are believed to be descended from a~single body, which got fractured in thousands of smaller bodies over several perihelia. The~evidence for this is the~strong similarity in the~orbital elements between the~individual Kreutz group comets \citep{jones2018science}, but the~exact origin story of the~group is not easily established \citep{kalinicheva2017specific,fernandez2021origin}. What is certain is that dust is released from the~comets of the~group in a~spatially highly non-uniform way. 

Meteor is a~visual phenomenon accompanying the~entry of a~sufficiently massive dust grain in the~Earth's atmosphere. Over a~$100$ distinct meteor showers were identified and confirmed to this day \citep{jenniskens2020removing}, which well document the~spatial inhomogeneity of the~dust in the~solar system. The~number of observed meteors, which belong to shower, is comparable to the~number of meteors which do not \citep{jenniskens2016cams}. 

Meteors are caused by comparably large grains of $r \gtrsim \, \si{mm}$ and these are very rare among in situ detections, which are dominated by $r \lesssim \, \si{\mu m}$ grains. These massive and sparse grains however produce smaller grains at collisions, which may be much more frequent where the~meteor stream crosses a~dense solar system dust cloud, and this may cause inhomogeneity even in the~flux of smaller dust. This effect was proposed as an explanation for the~post-perihelion enhancement of flux detected by PSP \citep{szalay2021collisional}. 

\subsubsection{Open questions}

The model, which we used to explain the~dust flux measured with SolO (Paper II), assumed a~symmetric and homogeneous dust cloud. This assumption might be checked statistically, for example by looking for unexplained variance in the~dust counts. One uncomplicated way of doing such analysis is comparing the~posterior predictive distribution to the~data, as we briefly did in Paper II, finding that the~used model was appropriate, and that no major contributor to the~flux was overlooked. 






