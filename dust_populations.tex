In principle, all dust grains are subjects to the same forces, as we discussed in the previous chapter. However, different forces are dominant for different grains, owing to their different sizes, states and  heliocentric locations. The history of a grain is also often of interest. For these reasons, it makes sense to discuss \textit{populations} of dust particles in the Solar system. As we'll see, investigating only dust grains belonging to a certain population at a time is possible, and reveals tangible information. 

\section{Bound dust}

Among the forces discussed in Ch. \ref{ch:forces}, the force with the steepest proportion to the grain's radius $r$ was gravity: $F_g \propto r^3$. Gravity therefore determines the motion of large bodies in the Solar system. The population of dust grains dominantly controlled by gravity is called \textit{bound dust}. We may encounter the term \textit{alpha} (abv. $\alpha$) meteoroids, but appropriateness of the term as a synonym of bound dust was questioned \citep{sommer2023alpha} and this term will not be used this way.

The size range of interest can be estimated by comparison with the forces dependent $\propto r^2$, namely radiation pressure force $F_{RP}$ and Poynting-Robertson drag $F_{PR}$. If the criterion for $F_{RP} \ll F_g$ is $\beta<0.1$, then using Eq. \ref{eq:beta_estimate} we find $r>10 \, \si{\mu m}$. The Poynting-Robertson time-scale of $r=10 \, \si{\mu m}$ dust (Eq. \ref{eq:PR_estimate}) is approximately $13 \, \si{kyr}$. For completeness, using Eqs. \ref{eq:charge_estimate} and \ref{eq:EM_estimate} we find that for $r=10 \, \si{\mu m}$ dust, the electromagnetic force $F_{EM} < 2\cdot 10^{-4} F_g$. Therefore, the motion of $r>10 \, \si{\mu m}$ dust is dominated by gravity.

Since much larger dust has a low ratio of surface to mass, and since much smaller dust does neither last long in orbit, nor scatters light very effectively, it is the range of $10 \, \si{\mu m} < r < 100 \, \si{\mu m}$ which contributes to the intensity of Zodiacal light the most \citep{leinert1981zodiacal}. 

mention collisonal balnce:  Dohnanyi 1969 and a simple explanation that the power law slope doesn't depend on the crushing law

mention open questions

\section{Beta meteoroids}

mention open questions

\section{Interstellar dust}

mention open questions

\section{Nano dust}

mention open questions

\section{Jovian dust}

volcanic activity

mention open questions (life on Enceladus?)

