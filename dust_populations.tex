Different sources of dust, along with different forces acting on dust grains as a result of their location and size, allow us to distinguish \textit{populations} of dust. In this chapter, we are going to discuss several important populations of dust in the Solar system.

\section{Bound dust}

Among the forces discussed in Ch. \ref{ch:forces}, the force with the steepest proportion to the grain's radius $r$ is gravity: $F_g \propto r^3$. Gravity therefore determines the motion of large bodies in the Solar system. The dust particles that are dominantly influenced by the gravity and are in bound orbits, are denoted as \textit{bound dust}. The term \textit{zodiacal dust} is sometimes used, since this dust contributes the most to the zodiacal light. We may encounter the term \textit{alpha} (abv. $\alpha$) meteoroids, sometimes used as interchangeably, but appropriateness of the term as a synonym to bound dust was debated \citep{sommer2023alpha}, since the term was originally used to describe highly eccentric bound grains only \citep{zook1975source} and we will not use the term this way.

\subsubsection{Size range}

The size range of interest can be estimated by comparison with the forces dependent $\propto r^2$, namely radiation pressure force $F_{RP}$ and Poynting-Robertson drag $F_{PR}$. If the criterion for $F_{RP} \ll F_g$ is $\beta<0.1$, then using Eq. \ref{eq:beta_estimate} we find $r>10 \, \si{\mu m}$. The Poynting-Robertson time-scale of $r=10 \, \si{\mu m}$ dust (Eq. \ref{eq:PR_estimate}) is approximately $13 \, \si{kyr}$. For completeness, using Eqs. \ref{eq:charge_estimate} and \ref{eq:EM_estimate} we find that for $r=10 \, \si{\mu m}$ dust, the electromagnetic force $F_{EM} < 2\cdot 10^{-4} F_g$. Therefore, the motion of $r>10 \, \si{\mu m}$ dust is dominated by gravity.

Since much larger dust has a low ratio of surface to mass, and since much smaller dust does neither last long in orbit, nor scatters light very effectively, it is the range of $10 \, \si{\mu m} < r < 100 \, \si{\mu m}$ which contributes to the intensity of Zodiacal light the most \citep{leinert1981zodiacal}. 

\subsubsection{Dynamics}

The spatial density of $10 - 100 \, \si{\mu m}$ dust near the ecliptic plane was established based on remote observations to approximately follow 
\begin{equation}
    n(r) \propto r^{-1.3}
\end{equation}
in the range of $0.3 \, \si{AU} < r < 1 \, \si{AU}$ \citep{leinert1981zodiacal}, and it is believed to be constantly replenished by fragments of colliding, larger, $r \geq 1 \, \si{mm}$ dust with spatial density of $\propto r^{-\nu}$, where $1 < \nu < 1.1$ \citep{leinert1983maintain}.

Bound dust grains are losing mass. This happens gradually, due to erosion, which faster, if the grains are closer to the Sun. This process is therefore accelerate by the shrinking of the orbital distance due to Poynting-Robertson drag. The grains also lose mass abruptly, in collisions, as we discussed in Ch. \ref{ch:forces}. It was found that collisions are responsible for the greatest proportion of mass loss from the bound dust population \citep{grun1985collisional}. On the other hand, the population is refilled by dust newly released from comets and in collisions between larger asteroids. Therefore, the bound dust population is believed to be in the state of dynamical balance. 

It was found \citep{dohnanyi1969collisional} that a steady-state mass distribution of asteroids under the influence of erosive and catastrophic collisions with each other (Sec. \ref{ch:collisions}) is established in the power-law form:
\begin{equation}
    f(m) \propto m^{-\alpha},
\end{equation}
where $\alpha = 11/6 \approx 1.83$, provided that the crushing-law (Eq. \ref{eq:crushing_law}) slope $\eta < 2$, that is, there are not too many large objects left after the collision. It is important to note that under the assumption that $\eta < 2$ and that the catastrophic collisions dominate the mass loss, the slope $\alpha = 11/6$ is not a function of $\eta$ at all, and is only a very shallow function of $\eta$ if the erosive collisions are relevant. It was also shown that the solution of $\alpha = 11/6$ is stable if perturbed by an additional inflow of low or high $m$ into the distribution \citep{dohnanyi1969collisional}. This does not hold fully for the dust grains with $r < \si{\mu m}$, since more loss processes are relevant in this region \citep{grun1985collisional}, most notably, $\beta$-meteoroid production.

\subsubsection{Open questions}

Remote observations provide information about the spatial distribution of dust, while in-situ detections and modelling provide information about its mass distribution. Both are however very insensitive to dust eccentricities and inclinations, therefore to the velocity distribution and the spatial distribution out of the ecliptic plane. While it is generally believed that the dust is concentrated around the ecliptic plane, it is not straightforward to deduce the off-ecliptic density dependence from remote observations and several models were proposed and debated \citep{giese1986three}.

Since the lifetime of dust grains in near vicinity of the Sun is very low due to intense erosion processes (Sec. \ref{ch:erosion}), a dust-free zone, enveloped in a dust-depletion zone was hypothesized \citep{russell1929meteoric}. Due to the limitations of the experimental techniques and the difficulties with decomposing the brightness measurements to dust and other light-producing phenomena, the dust free zone was not convincingly observed to this date, although the dust depletion zone was from onboard Parker Solar Probe \cite{stenborg2018characterization}. It was recently estimated, based on remote observation from Parker Solar Probe, that the dust depletion zone lies between $5 \, R_{sun}$ and $19 \, R_{sun}$, while the dust free zone is expected inward of $5 \, R_{sun}$ \citep{stenborg2022psp}. This implies that the Parker Solar Probe has already travelled well into the dust depletion zone with its perihelia of $\approx 12 \, R_{Sun}$.

\section{Beta meteoroids}

Pioneer 8 and 9 spacecraft discovered a previously unobserved population of dust grains, when it was apparent that most of the grains of the size $r < 10^{-6} \, \si{m}$ were coming from the direction of the Sun \cite{berg1973evidence}. This phenomenon is explained as a population of dust leaving the Sun's gravity well on a hyperbolic trajectory due to the solar radiation pressure successfully competing with solar gravity, and the term \textit{beta meteoroids} was coined as to name the population \cite{zook1975source}. Many other space missions observed the population since, for example \cite{zaslavsky2012interplanetary,malaspina2014interplanetary,zaslavsky2021first}.

\subsubsection{Size range}

As we showed previously (Eq. \ref{eq:effective_gravity}), the laws of motion do not change even if the radiation pressure is significant, it is the effective gravitational parameter $\mu_{e} = (1-\beta) \mu$, which changes with $\beta$. Since the escape speed $v_e$ and the circular speed $v_c$ differ by a factor of $\sqrt{2}$ and both depend on $\mu$ (Eqs. \ref{eq:circular_speed}, \ref{eq:escape_speed}), in terms of effective gravity:
\begin{equation}
    v_c(\beta=0) = v_e(\beta=1/2).
\end{equation}
This implies that if a dust grain on a circular orbit with $\beta=0$ suddenly changes its $\beta$ value to $\beta = 1/2$, it becomes critically unbound, that is on parabolic trajectory. A change to a higher $\beta$ would then naturally lead to a hyperbolic trajectory. Since dust grains change their $\beta$ value suddenly at collisions, and since the post-collision fragment speed is not going to be very different from the pre-collision parent object speed, the value of $\beta=1/2$ is often considered the minimum $\beta$ required for the dust grain to be on an unbound trajectory pointed away from the Sun.

Using our estimate with simplistic assumptions (Eq. \ref{eq:beta_estimate}), $\beta > 0.5$ for $r < 1.9 \, \si{\mu m}$. More refined estimates \citep{kimura2003composition} point to the region of $100 \, \si{nm}$ to $1 \, \si{\mu m}$ and are material dependent. We note that if the parent grains are on eccentric orbits, the requirement of $\beta > 0.5$ is not exact, but this doesn't influence the size estimate greatly. We also note that beta meteoroids are smaller than those, which contribute to the brightness of the Zodiacal light the most. This makes remote observation difficult, and they are therefore, in practice, only detected in-situ.

\subsubsection{Dynamics}

The beta meteoroids are believed to be created in collisions between bound grains, which naturally happens where the bound dust spatial density is high. It was recently reported that the dust detections of Parker Solar Probe are compatible with the beta meteoroid creation region at around $10 - 20 \, R_{Sun}$ \citep{szalay2021collisional}. Since beta meteoroids are unbound, they leave the inner Solar system shortly after their creation, and other forces have limited time to act. For example, even relatively short Poynting-Robertson lifetime of $\approx \, kyr$ is unimportant compared to the timescale of $< 1 \, \si{yr}$, in which a grain created in the vicinity of the Sun passes beyond $1 \, \si{AU}$. On their way out, beta meteoroids follow the conservation of angular momentum and the conservation of energy. The former implies that sufficiently far from the region of their creation, their velocity is nearly purely radial. The latter implies they accelerate if $\beta > 1$, since they feel net solar repulsion, and they decelerate if $\beta < 1$, since they feel net solar attraction. In Paper II of this thesis, we found that the beta meteoroids detected with Solar Orbiter decelerate significantly, which implies effective mean $\beta$ near the liberation threshold, that is $\beta \approx 0.5$. 

\subsubsection{Open questions}

One of the open questions related to beta meteoroids is to what extent is their flux constant in time and rotationally symmetric. Answering this question is complicated, since beta meteoroids are only detected in-situ, and there is always a bond between the time and location of the detecting spacecraft. 

Beta meteoroids were claimed to be produced in collisions between the main, rotationally symmetric bound dust population and the \textit{geminids} meteoric stream \citep{szalay2021collisional}, as a means to explain excess detections with Parker Solar Probe. Further investigation of this phenomenon is desirable.

\section{Interstellar dust}

We know that dust of various sizes is present in the galactic interstellar medium of our galaxy, since it is needed in order to explain the interstellar extinction measurements \citep{desert1990interstellar}. The heliosphere moves with respect to the local interstellar medium, and the relative speed and direction is known, since the velocity distribution of interstellar neutral gas was measured, for example onboard Ulysses spacecraft \citep{witte2004kinetic}. It was also with the Ulysses spacecraft, that a population of dust, seemingly coming from the direction of the interstellar neutral gas was detected \citep{grun1993discovery}. Since then, other spacecraft reported interstellar dust (\textit{ISD}) detections in-situ \citep{zaslavsky2012interplanetary,malaspina2014interplanetary}. 

\subsubsection{Size range and dynamics}

ISD is created by condensation and aggregation, and destroyed by sputtering, sublimation and collisions \citep{mann2010interstellar}. Only the ISD grains with masses higher than $3\cdot 10^{-16} \, \si{kg}$ (that is $r \approx 0.7 \, \si{\mu m} $) are believed to be able to enter the heliosphere \citep{kimura1998electric}. Upon entry, the grains move according to the laws of gravity and under the influence of Lorenz force. The effective gravitational parameter $\mu_e(\beta)$ (see Eq. \ref{eq:effective_gravity}) depends on the amount of radiation pressure compared to gravity. Should $\beta > 1$, the grains are deflected and do not reach close vicinity of the Sun \citep{henriksen2022interstellar}. If $\beta \approx 1$, they do not feel the influence of the Sun and move with nearly constant speed. The motion of the dust grains is also influenced by the Lorentz force, and this effect is also size dependent \cite{morfill1979motion}.

\subsection{Open questions}

The exact dynamics of ISD grains in the Solar system is an object of study, especially the possible effect of dust \textit{focusing} and \textit{defocusing} to and from the plane of ecliptics due to the polarity of the interplanetary magnetic field, which switches between N-S and S-N configurations with the period of 22 years, that is two solar cycles \citep{morfill1979motion}. This long period, longer than the duration of many experiments, makes the effect difficult to study. It is however the case that most of the in-situ ISD detections happened around the year 2009, during the solar minimum between the solar cycles 23 and 24, when the interplanetary magnetic field was in the \textit{focusing} configuration \citep{babic2022situ} and the observed ISD flux decreased significantly, although not disappear completely, since then. It was hypothesized that the change in observed flux is physical and that the flux should rise again during the minimum between the solar cycles 25 and 26, which is expected in the late 2020's \citep{mann2010interstellar}. 

Gravitational focusing behind the Sun of the ISD grains with $\beta < 1$ and an increased spatial density in this region is a logical consequence of their trajectories, yet experimental evidence of this is not available \citep{mann2010interstellar}. The low density, the temporal variability and the difficulty of distinguishing this population of dust from beta meteoroids so far prevented the detection.

\section{Nanodust}

Small dust grains are produced in collisions of larger dust. If the created dust grains are so small that $\beta \ll 1$, the radiation pressure does not liberate them from the gravity well of the Sun. Small grains have however high capacitance to mass ratio $C/m$, and by extension, high charge to mass ratio $q/m$. For example $r = 10 \, \si{nm}$ spherical grain at the potential of $\phi = 1 \, \si{V}$ has $q/m \approx 71 \, \si{Ckg^{-1}}$ (Eq. \ref{eq:charge_estimate}). These grains are therefore highly susceptible to be influenced by Lorentz force \citep{czechowski2010formation}. 

As we discussed in Sec. \ref{ch:solar_wind_pressure}, solar wind pressure provides an additional pseudo-Poynting-Robertson drag, which is usually smaller than radiational Poynting-Robertson drag for larger dust, but is likely very relevant for small dust with $r<10\, \si{nm}$ \citep{mukai1982solar}, when even radial solar wind pressure might contest gravity. 

Planetary nanodust was also identified in Cassini/RPWS spectra 
at $\SI{1}{AU}$ \citep{schippers2014nanodust}, and  between $\SI{1}{AU}$ and $\SI{5}{AU}$, and tt was concluded, that the asteroid belt's contribution to the nanodust flux is negligible \citep{schippers2015nanodust}. It was also identified in the jovian system \citep{meyer2009detecting}, and in the saturnian system \citep{kempf2005high}. It was in fact concluded that nanodust is so ubiquitous, that some was detected whenever the RPWS instrument was on \citep{schippers2015nanodust}. Cometary nanodust was detected by Giotto/PIA and Vega/PUMA mass spectrometers net the comet Halley, although very limited information about their composition was yielded, due to low signal \cite{utterback1990attogram}.

\subsection{Open questions}

Since nanodust grains are highly susceptible to the influence of Lorentz force, a strong temporal variation is naturally expected, as the inperplanetary magnetic field is not constant \citep{poppe2020effects}. The complications are that nanodust isn't observed remotely, and observation in-situ is complicated \citep{pantellini2012nano,kellogg2016dust,kellogg2017note}. 

It was reportedly detected on Solar Terrestrial Relations Observatory (\textit{STEREO}) spacecraft \citep{meyer2009dust} and mostly disappear after the solar cycle 24 started after 2010 \citep{zaslavsky2012interplanetary}. It was argued that this might be due to an unfavourable interplanetary magnetic field orientation during the solar cycle 24, and that the nanodust flux will reappear in the STEREO measurements later during the solar cycle 25, at some time before 2028 \citep{poppe2022effects}.

\section{Localized dust}

\subsection{Inhomogeneity}

Unlike the omnipresent gradual erosion, the inflow of new dust into the system is very stochastic and non-constant, as for example comets, which are believed to be an important source of the dust cloud, are not uniformly distributed in time and space.

More than a half of all the catalogued comets are so-called \textit{sungrazers}, which are comets which have the perihelion in a close vicinity of the Sun \citep{jones2018science}, at the heliocentric distance of a few solar radii $R_{Sun}$. These are typically small objects ($r<100 \, si{m}$) which don't survive the passage, but are destroyed near the perihelion by the combination of heat and tidal waves. Their material is then partially transferred to the dust cloud. In principle, such highly eccentric comets might arrive from any direction, but a few massive bodies were identified to have been destroyed in the past, which are responsible for most of the identified near-sun comets \citep{jones2018science}. Among these, the most prominent cometary groups is the \textit{Kreutz group} \citep{kreutz1888untersuchungen}, members of which are believed to be descended from a single body, which got fractured in thousands of smaller bodies over several perihelia. The evidence for this is the strong similarity in the orbital elements between the individual Kreutz group comets \citep{jones2018science}, but the exact origin story of the group is not easily established \citep{kalinicheva2017specific,fernandez2021origin}. What is certain is that dust is released from the comets of the group in a spatially highly non-uniform way. 

Meteor is a visual phenomenon accompanying the entry of a sufficiently massive dust grain in the Earth's atmosphere. Over a $100$ distinct meteor showers were identified and confirmed to this day \citep{jenniskens2020removing}, which well document the spatial inhomogeneity of the dust in the Solar system. The number of observed meteors, which belong to shower, is comparable to the number of meteors which do not \citep{jenniskens2016cams}. 

Meteors are caused by comparably large grains of $r \gtrsim \, \si{mm}$ and these are very rare among in situ detections, which are dominated by $r \lesssim \, \si{\mu m}$ grains. These massive and sparse grains however produce smaller grains at collisions, which may be much more frequent where the meteor stream crosses a dense Solar system dust cloud, and this may cause inhomogeneity even in the flux of smaller dust. This effect was proposed as an explanation for the post-perihelion enhancement of flux detected by Parker Solar Probe \citep{szalay2021collisional}. 

\subsection{Planetary dust}

The dust directly linked to a planet, is denoted as \textit{planetary dust}. The passage of Pioneers 10 and 11 through the jovian system has discovered flux of dust several orders of magnitude higher than the flux commonly observed elsewhere at the same heliocentric distance \citep{humes1974interplanetary}. The subsequent study by Voyager, which revealed the active volcanism at Io \citep{kruger2004jovian}, and Ulysses, which measured intermittent dust streams origination in the jovian system \citep{grun1993discovery}, confirmed the jovian system as a locally important source of dust. Cassini detected nanodust near Jupiter, which was also confirmed to be originationg at the moon Io \citep{meyer2009detecting}. Similarly in the saturnian system, Enceladus was linked to the tenuous E-ring of Saturn \citep{baum1981saturn}. Cassini data also shown nanodust detections \citep{kempf2005high} and later confirmed the volcanic activity on Enceladus \citep{spahn2006cassini}, which feeds the E-ring of Saturn \citep{kempf2010enceladus}. 

