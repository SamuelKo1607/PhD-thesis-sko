% Outline

Rocky, icy, and metallic objects in space, smaller than asteroids, are called cosmic dust. Cosmic dust is created as a debris of collisions of larger objects, but also by condensation of gaseous phase, or by being expelled from a larger body, such as a comet or a moon with active volcanism. Cosmic dust, which originates in the solar system is called interplanetary dust, as opposed to cosmic dust which occupies the interstellar or intergalactic space. The circumsolar interplanetary dust is responsible for zodiacal light, which is the diffuse glow observed in post-sunset and pre-sunrise night sky near the ecliptic plane.

Interplanetary dust cloud is an integral part of the solar system, and its dynamics tells about the past and the present of other bodies. In the inner solar system, it is a probe into the vicinity of the Sun. The dust, which gets too close to the Sun, is destroyed by the extreme near-solar conditions, forming a dust-free zone around the Sun. By understanding how the circumsolar dust moves, where the collisions happen, where it is destroyed, and how its characteristics depend on its composition, we find more about the conditions around the Sun, which have influence on the rest of the solar system. For example, When a dust grain is sputtered or evaporated by the Sun, its material enters the solar wind, and is carried out of the inner solar system. With understanding of the dust dynamics in the solar system, the study of other stars' circumstellar dust clouds and planetary disks is also enabled.

The atmosphere of Earth offers a great target for cosmic dust, and the cosmic dust entering the atmosphere is observable in the form of meteors. Vast majority of meteors is of interplanetary origin. The amount of dust entering the atmosphere was estimated decades ago, and remains fairly uncertain. 

When spacecraft move through a dusty environment, they randomly collide with dust grains of the dust cloud. Little can be told about the cloud based on a single collision, but having observed many collisions, generalizations can be made about their size and speed distribution within the cloud. Even more can be told by comparing a theoretical framework for where the dust comes from and how it moves, to the measurement. 

In this thesis, we explore the measurements of two spacecraft: Solar Orbiter (SolO) and Parker Solar Probe (PSP). These are both Sun-orbiting spacecraft, and are the two human made objects which venture the closest to the Sun. Using their electrical antennas to register collisions with dust grains, they provide dust measurements from the region where no other spacecraft measured before. This alone makes their measurements very valuable, and the fact that they operate simultaneously and over several years only more so.

In order to understand the dynamics in the interplanetary dust cloud, we make use of their similarities and differences. Different statistical modeling approaches must be taken to yield the maximum information in different regions. In the region further away from the Sun, where collisions with dust grains are relatively rare, reliable and consistent detection, in combination with counting statistics, reveals the spatial distribution. In the close vicinity of the Sun, several effects happen at the same time and by comparing the measurements to models, we identify the dust properties, which are the most important for the description of the cloud. By aapplying the appropriate tools, we find more about the dust grains' speed, masses, collisions, and location in different regions in the inner solar system.

% Outline

The aim of the thesis is to yield information on the interplanetary dust inside Earth's orbit from electrical antenna measurements performed by two unique spacecraft, SolO and PSP, using more mathematically precise approach than what was used previously, wherever possible. Single-grain dust properties, as well as forces acting on individual dust grains are introduced in Ch. \ref{ch:forces}. Dust grains in the Solar system compose the interplanetary dust cloud, in which grains of similar properties follow similar trajectories, forming individual dust populations, which are introduced in Ch. \ref{ch:populations}. To yield the most information possible, sharp statistical tools are needed, some of which are introduced in Ch. \ref{ch:statistics}. The dust detection principles are described in Ch. \ref{ch:detection}, where emphasis is put on antenna measurements. The articles, which make up a part of this thesis, are described in Ch. \ref{ch:sum-paper}. Finally, in Ch. \ref{ch:conclusion}, we conclude and offer outlook.