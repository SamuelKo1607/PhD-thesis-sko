\section{Plain language summary}

Rocky, icy, and metallic objects in space, smaller than asteroids, are called cosmic dust. Cosmic dust which originates in the solar system is called interplanetary dust, as opposed to cosmic dust which occupies the interstellar or intergalactic space. Cosmic dust is created as a debris of collisions of larger objects, but also by condensation of gaseous phase, or by being expelled from a larger body, such as a comet or a moon with active volcanism. It is responsible for the planetary rings, but also for the the zodiacal light observed in the post-sunset and pre-sunrise night sky. The atmosphere of Earth offers a great target cosmic dust, and the cosmic dust entering the atmosphere is observable in the form of meteors. 

When spacecraft move through a dusty environment, they collide with dust grains in a random fashion. Little can be told about the dust cloud based on the fact that a single collision happened, but having observed many collisions, generalizations can be made about their size and speed distribution within the cloud. Even more can be told, if a theoretical framework for where the dost comes from and how it moves, is available. 

Cosmic dust is an integral part of the solar system, and its dynamics tells about the past and present of other bodies. For example, the structure of the rings of Saturn tells about the past and present of the saturnian system. Similarly, the dust around the Sun is a probe into the vicinity of the Sun. By understanding how the circumsolar dust moves, where the collisions happen, where it is destroyed, and how its characteristics depend on its composition, we find more about the conditions around the Sun. 

We explore the measurements of two spacecraft: Solar Orbiter and Parker Solar Probe. These are Sun-orbiting spacecraft, which use their electrical antennas to register collisions with dust grains. In order to understand the dynamics in the interplanetary dust cloud, we make use of their similarities and differences alike. Different statistical modeling approach must be taken to yield the maximum information, and in doing so, we find more about the dust grains' speed, masses, collisions, and location is space.

\section{Outline}

The thesis is focused on the topic of yielding information on the interplanetary dust from electrical antenna measurements. Single-grain dust properties, as well as forces acting on individual dust grains are introduced in Ch. \ref{ch:forces}. The dust grains in the Solar system compose the interplanetary dust cloud, in which grains of similar properties follow similar trajectories, forming individual dust populations, which are introduced in Ch. \ref{ch:populations}. To yield the most information possible, sharp statistical tools are needed, some of which are introduced in Ch. \ref{ch:statistics}. The dust detection principles are described in Ch. \ref{ch:detection}, where emphasis is put on antenna measurements. The papers, which make up a part of this thesis, are described in Ch. \ref{ch:sum-paper}. Finally, in Ch. \ref{ch:conclusion}, we conclude and offer outlook.