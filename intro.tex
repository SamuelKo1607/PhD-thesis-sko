Keep it short (for my parants-ish). 

- a brief overview of dust: young cometary dust (citation) tbd, icy dust od enceladus, collisionally produced dust

- dust as a probe into corona
- dust study of Enceladus
- dust into the atmoshpere

- the focus is to understand what dust is out there, how it behaves and how it is measureable




\section{Outline}

The thesis is focused on the topic of yielding information on the interplanetary dust from electrical antenna measurements. Single-grain dust properties, as well as forces acting on individual dust grains are introduced in Ch. \ref{ch:forces}. The dust grains in the Solar system compose the interplanetary dust cloud, in which grains of similar properties follow similar trajectories, forming individual dust populations, which are introduced in Ch. \ref{ch:populations}. To yield the most information possible, sharp statistical tools are needed, some of which are introduced in Ch. \ref{ch:statistics}. The dust detection principles are described in Ch. \ref{ch:detection}, where emphasis is put on antenna measurements. The papers, which make up a part of this thesis, are described in Ch. \ref{ch:sum-paper}. Finally, in Ch. \ref{ch:conclusion}, we conclude and offer outlook.